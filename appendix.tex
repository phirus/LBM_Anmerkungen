\chapter{Anhang}
% \section{Relaxationsmatrix}
% \label{anhang:relaxationsmatrix}
% \rotatebox{90}{
% $ S = \left( \begin{smallmatrix}
% \frac{4 s_3+4 s_2}{9} & -\frac{2 s_3-s_2}{9}                 & \frac{s_3-2 s_2}{9}                 & -\frac{2 s_3-s_2}{9}                 & \frac{s_3-2 s_2}{9}                 & -\frac{2 s_3-s_2}{9}                 & \frac{s_3-2 s_2}{9}                 & -\frac{2 s_3-s_2}{9}                 & \frac{s_3-2 s_2}{9} \\
% -\frac{2 s_3-s_2}{9}  & \frac{9 \omega+12 s_5+4 s_3+s_2}{36} & -\frac{3 s_5+s_3+s_2}{18}           & -\frac{9 \omega-4 s_3-s_2}{36}       & \frac{3 s_5-s_3-s_2}{18}            & \frac{9 \omega-12 s_5+4 s_3+s2}{36}  & \frac{3 s_5-s_3-s_2}{18}            & -\frac{9 \omega-4 s_3-s_2}{36}       & -\frac{3 s_5+s_3+s_2}{18} \\
% \frac{s_3-2 s_2}{9}   & -\frac{3 s_5+s_3+s_2}{18}            & \frac{9 \omega+6 s_5+s_3+4 s_2}{36} & -\frac{3 s_5+s_3+s_2}{18}            & -\frac{9 \omega-s_3-4 s_2}{36}      & \frac{3 s_5-s_3-s_2}{18}             & \frac{9 \omega-6 s_5+s_3+4 s_2}{36} & \frac{3 s_5-s_3-s_2}{18}             & -\frac{9 \omega-s_3-4 s_2}{36} \\
% -\frac{2 s_3-s_2}{9}  & -\frac{9 \omega-4 s_3-s_2}{36}       & -\frac{3 s_5+s_3+s_2}{18}           & \frac{9 \omega+12 s_5+4 s_3+s_2}{36} & -\frac{3 s_5+s_3+s_2}{18}           & -\frac{9 \omega-4 s_3-s_2}{36}       & \frac{3 s_5-s_3-s_2}{18}            & \frac{9 \omega-12 s_5+4 s_3+s_2}{36} & \frac{3 s_5-s_3-s_2}{18} \\
% \frac{s_3-2 s_2}{9}   & \frac{3 s_5-s_3-s_2}{18}             & -\frac{9 \omega-s_3-4 s_2}{36}      & -\frac{3 s_5+s_3+s_2}{18}            & \frac{9 \omega+6 s_5+s_3+4 s_2}{36} & -\frac{3 s_5+s_3+s_2}{18}            & -\frac{9 \omega-s_3-4 s_2}{36}      & \frac{3 s_5-s_3-s_2}{18}             & \frac{9 \omega-6 s_5+s_3+4 s_2}{36} \\
% -\frac{2 s_3-s_2}{9}  & \frac{9 \omega-12 s_5+4 s_3+s_2}{36} & \frac{3 s_5-s_3-s_2}{18}            & -\frac{9 \omega -4 s_3-s_2}{36}      & -\frac{3 s_5+s_3+s_2}{18}           & \frac{9 \omega+12 s_5+4 s_3+s_2}{36} & -\frac{3 s_5+s_3+s_2}{18}           & -\frac{9 \omega-4 s_3-s_2}{36}       & \frac{3 s_5-s_3-s_2}{18} \\
% \frac{s_3-2 s_2}{9}   & \frac{3 s_5-s_3-s_2}{18}             & \frac{9 \omega-6 s_5+s_3+4 s_2}{36} & \frac{3 s_5-s_3-s_2}{18}             & -\frac{9 \omega-s_3-4 s_2}{36}      & -\frac{3 s_5+s_3+s_2}{18}            & \frac{9 \omega+6 s_5+s_3+4 s_2}{36} & -\frac{3 s_5+s_3+s_2}{18}            & -\frac{9 \omega-s_3-4 s_2}{36} \\
% -\frac{2 s_3-s_2}{9}  & -\frac{9 \omega-4 s_3-s_2}{36}       & \frac{3 s_5-s_3-s_2}{18}            & \frac{9 \omega-12 s_5+4 s_3+s_2}{36} & \frac{3 s_5-s_3-s_2}{18}            & -\frac{9 \omega-4 s_3-s_2}{36}       & -\frac{3 s_5+s_3+s_2}{18}           & \frac{9 \omega+12 s_5+4 s_3+s_2}{36} & -\frac{3 s_5+s_3+s_2}{18} \\
% \frac{s_3-2 s_2}{9}   &  -\frac{3 s_5+s_3+s_2}{18}           & -\frac{9 \omega-s_3-4 s_2}{36}      & \frac{3 s_5-s_3-s_2}{18}             & \frac{9 \omega-6 s_5+s_3+4 s_2}{36} & \frac{3 s_5-s_3-s_2}{18}             & -\frac{9 \omega-s_3-4 s_2}{36}      & -\frac{3 s_5+s_3+s_2}{18}            & \frac{9 \omega+6 s_5+s_3+4 s_2}{36}
% \end{smallmatrix} \right) $
% }
\section{Volumenviskosität}
In der Relaxationsmatrix hängt der Parameter $s_2$ direkt von der Volumenviskosität (\emph{bulk viscosity}, \emph{second viscosity}) ab \cite{Guo2013}:
\begin{equation}
 \nu' = \frac{c^2}{3} \left( \frac{1}{s_2} - \frac{1}{2} \right)  \delta t.
\end{equation}

Die Volumenviskosität ist ein Maß für die innere Reibung bei Kompression. 
Bei Beachtung der sog. zweiten Viskosität $\nu'$ ergibt sich folgender Spannungstensor \cite{Dellar2001}:
\begin{equation}
\label{eq:Spannungsmatrix}
 \mathbb{S} = \sigma_{\alpha \beta} = \mu \left( \partial_{\alpha} u_{\beta} + \partial_{\beta} u_{\alpha} - \frac{2}{3} \delta_{\alpha \beta} \nabla \cdot \vec{u} \right) + \mu' \delta_{\alpha \beta} \nabla \cdot \vec{u}
\end{equation}
(Bei Inkompressibilität wird $\nabla \cdot \vec{u} = 0$.)
Im englischsprachigen Raum wird außerdem oft $K = \mu' + \frac{2}{3} \mu$ als \emph{bulk viscosity} bezeichnet, was mitunter zu Verwirrung führt \cite{Rosenhead1954}.
Hier wird die bei Lattice Boltzmann Anwendungen übliche erste Benamung verwendet.
Analog zur Scherviskosität gitl: $\mu' = \nu' \cdots \rho$.

\subsection{Wahl des Wertes}
Für die Simulation muss letztlich ein Wert für $\zeta$ gewählt werden. 
Mit der Stokes-Hypothese wird für dünne einatomige Gase \cite{Graves1999}:
\begin{equation}
 0 = \mu' + \frac{2}{3} \mu.
\end{equation}
Die Volumenviskosität wäre damit $\mu' = - \frac{2}{3} \mu$. Da diese Hypothese aber nur selten gilt, wird häufig auf experimntelle Werte zurück gegriffen.

Hier wird der Wert $\mu' / \mu \approx 2$ für Wasser \cite{Rosenhead1954} verwendet.

\subsection{Berechnung des Druckes}
Aus der Spannungsmatrix \eqref{eq:Spannungsmatrix} kann außerdem der mechanische Druck im Fluid bestimmt werden.
Dazu wird die durchschnittliche Normal-Spannung verwendet \cite{Graves1999}:
\begin{equation}
 p_{\text{mech}} = \frac{\sigma_{xx} \cdot \sigma_{yy} \cdot \sigma_{zz}}{3}
\end{equation}

Dieser Druck sollte nicht mit dem thermodynamischen Druck ($p_{\text{th}}=c^2 \rho$) verwechselt werden. Bei Anwendung der Stokes Hypothese fallen diese beiden Drücke zusammen \cite{Graves1999}.

Die partiellen Ableitungen von $u$ können jeweils durch einen mittleren Differenzenquotienten angenähert werden:
\begin{equation}
 \left. \frac{\partial u(x)}{\partial x}\right|_{x_0} = \frac{u(x_0 + \Delta x) - u(x_0 - \Delta x)}{2 \Delta x}
\end{equation}

Wird außerdem statt dem MRT-Schema ein BGK-Schema verwendet wird, ist $\mu' =  \mu$.
